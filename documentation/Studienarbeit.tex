%Option twoside zur Optimierung für beidseitigen Druck
\documentclass[12pt,ngerman,numbers=noenddot,abstract=true,version=first,headsepline]{scrreprt}
\renewcommand{\familydefault}{\sfdefault}
\usepackage[T1]{fontenc}
\usepackage[utf8]{inputenc}
\usepackage{geometry}
\geometry{verbose,tmargin=2.5cm,bmargin=2.5cm,head=35pt}
\setlength{\parskip}{\medskipamount}
\setlength{\parindent}{0pt}
\usepackage{array}
\usepackage{float}
\usepackage{textcomp}
\usepackage{multirow}
\usepackage{amsmath}
\usepackage{amsthm}
\usepackage{graphicx}
\usepackage{setspace}
\usepackage{microtype}
\usepackage{nomencl}
\usepackage[style=super, nopostdot, nonumberlist, nogroupskip, acronyms]{glossaries}

\makeglossaries 
\newglossaryentry{latex}
{
    name=LaTeX,
    description= {Markuplanguage für die Erstellung von wissenschaftlichen Berichten}
}
\newacronym{mes}{MES}{Manufacturing Execution System}
\newacronym{emes}{E-MES}{Eisenmann Manufacturing Execution System}
\makenomenclature

\setstretch{1.2}

\makeatletter

% verschieden Symbole, Zeichen wie (c), €
\usepackage{textcomp,units}

% Mehr Platz zwischen Tabelle und Untertitel
\usepackage{caption}
\captionsetup[table]{skip=10pt}

%Kapitelzahl sehr groß
\makeatletter% siehe De-TeX-FAQ 
    \renewcommand*{\chapterformat}{% 
    \begingroup% damit \unitlength-Änderung lokal bleibt 
    \setlength{\unitlength}{1mm}% 
    \begin{picture}(10,10)(0,5) 
        \setlength{\fboxsep}{0pt} 
        %\put(0,0){\framebox(20,40){}}% 
        %\put(0,20){\makebox(20,20){\rule{20\unitlength}{20\unitlength}}}% 
        \put(10,15){\line(1,0){\dimexpr 
            \textwidth-20\unitlength\relax\@gobble}}% 
        \put(0,0){\makebox(10,20)[r]{% 
            \fontsize{28\unitlength}{28\unitlength}\selectfont\thechapter 
            \kern-.05em% Ziffer in der Zeichenzelle nach rechts schieben 
            }}% 
        \put(10,15){\makebox(\dimexpr 
            \textwidth-20\unitlength\relax\@gobble,\ht\strutbox\@gobble)[l]{% 
            \ \normalsize\color{black}\chapapp~\thechapter\autodot 
            }}% 
        \end{picture} % <-- Leerzeichen ist hier beabsichtigt! 
    \endgroup 
}

\usepackage{ %a4wide,
            ellipsis, mparhack, %Fehlerkorrektur für Marginalien
            booktabs, longtable %schönere Tabellen
}


%Kurzfassung und Abstract (englisch) auf eine Seite
\renewenvironment{abstract}{
    \@beginparpenalty\@lowpenalty
    \begin{center}
    \normalfont\sectfont\nobreak\abstractname
    \@endparpenalty\@M
    \end{center}
}{
    \par
}

% schönerer Blocksatz!!
\usepackage{ifpdf} % part of the hyperref bundle
\ifpdf % if pdflatex is used 

%set fonts for nicer pdf view
\IfFileExists{lmodern.sty}{\usepackage{lmodern}}
    {\usepackage[scaled=0.92]{helvet}
    \usepackage{mathptmx}
    \usepackage{courier} }
\fi

% the pages of the TOC are numbered roman
% and a pdf-bookmark for the TOC is added
\pagenumbering{roman}
\let\myTOC\tableofcontents
\renewcommand\tableofcontents{
%\pdfbookmark[1]{Contents}{}
\myTOC
\clearpage
\pagenumbering{arabic}}

%Bezeichungen anpassen
%Babelpaket muß zuvor geladen werden
\usepackage[ngerman]{babel}
\addto\captionsngerman{ 
    \renewcommand{\figurename}{Abb.}% 
    \renewcommand{\tablename}{Tab.}% 
    \renewcommand{\abstractname}{Kurzfassung}
    \renewcommand{\nomname}{Abkürzungsverzeichnis}
}

%mehr Platz zwischen Überschrift und Tabelle
\newcommand{\@ldtable}{}
    \let\@ldtable\table
\renewcommand{\table}{ %
    \setlength{\@tempdima}{\abovecaptionskip} %
    \setlength{\abovecaptionskip}{\belowcaptionskip} %
    \setlength{\belowcaptionskip}{\@tempdima} %
    \@ldtable}

%Config für Programmcode
\usepackage{listings}
\usepackage{color}
\usepackage{scrhack}

\renewcommand{\lstlistlistingname}{Programm-Listings}

\definecolor{dkgreen}{rgb}{0,0.6,0}
\definecolor{gray}{rgb}{0.5,0.5,0.5}
\definecolor{mauve}{rgb}{0.58,0,0.82}

\lstset{frame=tb,
    language=Java,
    aboveskip=3mm,
    belowskip=3mm,
    showstringspaces=false,
    columns=flexible,
    basicstyle={\footnotesize\ttfamily},
    numbers=left,
    numberstyle=\tiny\color{gray},
    keywordstyle=\color{blue},
    commentstyle=\color{dkgreen},
    stringstyle=\color{mauve},
    breaklines=true,
    breakatwhitespace=true,
    tabsize=3
}

\AtBeginDocument{
    \def\labelitemiii{\(\circ\)}
}


%Hier können zusätzliche Befehle definiert werden
%Author, Title, Subject, Keywords müssen vorhanden sein

\renewcommand{\author}{Moris Kotsch}
\renewcommand{\title}{Titel der Arbeit}
\renewcommand{\subject}{Thema}
\newcommand{\keywords}{Stichworte, mit, Kommmata, getrennt}

\usepackage[automark,autooneside=false]{scrlayer-scrpage}
\clearscrheadfoot
\if@twoside
    \ofoot[\pagemark]{\pagemark}
    \ohead{\headmark}
\else
    \cfoot[\pagemark]{\pagemark}
    \ohead{
        \ifnum\value{section}>0
        \rightmark
        \fi
    }
    \ihead{
        \leftmark
    }
\fi

\makeatother

% Alle Querverweise und URLs als Link darstellen
% In der PDF-Ausgabe
\usepackage[colorlinks=true, bookmarks, bookmarksnumbered, bookmarksopen, bookmarksopenlevel=1,
    linkcolor=black, citecolor=black, urlcolor=blue, filecolor=blue,
    pdfpagelayout=OneColumn, pdfnewwindow=true,
    pdfstartview=XYZ, plainpages=false, pdfpagelabels,
    pdfauthor={\author{}}, pdftex,
    pdftitle={\title{}},
    pdfsubject={\subject{}},
    pdfkeywords={\keywords{}}]{hyperref}

\usepackage{babel}
\begin{document}
    \pagestyle{plain}
    \titlepage
\begin{center}
    \textbf{\large{}Duale Hochschule Baden-Württemberg }{\large\par}
    \par
\end{center}
\begin{center}
    \textbf{\large{}Stuttgart Campus Horb}{\large\par}
    \par
\end{center}
\begin{center}
    \begin{tabular}{l||r}
        \multicolumn{2}{c}{\vspace{3cm}}
        \tabularnewline
        \multicolumn{2}{c}{\includegraphics[height=3.5cm]{images/dhbwlogo}}
        \tabularnewline
        \multicolumn{2}{c}{}
        \tabularnewline
        %Firmenlogo 
        %\multicolumn{2}{c}{\includegraphics[scale=0.1]{images/ENisco_logo_solo}}
        %Platzhalter statt Firmenlogo
        \multicolumn{2}{c}{\vspace{59px}}
        \tabularnewline
    \end{tabular}
    \par
\end{center}
\vspace{2cm}

\begin{flushleft}
    \textbf{\Large{}\title{}}{\Large\par}
    \par
\end{flushleft}

\begin{flushleft}
    \textbf{\textit{Art der Arbeit: Studienarbeit}}
    \par
\end{flushleft}

\begin{flushleft}
    {\Large{}\rule[0.5ex]{1\columnwidth}{1pt}}{\Large\par}
    \par
\end{flushleft}

\begin{tabular}{ll}
    eingereicht von:\hspace{1cm} & \author{}
    \tabularnewline
    Matrikelnummer: & 1234567
    \tabularnewline
    Kurs: & TINF2018
    \tabularnewline
    Studiengang: & Informatik
    \tabularnewline
    Hochschule: & DHBW Stuttgart Campus Horb
    \tabularnewline
    Ausbildungsfirma: & ENisco by Forcam GmbH
    \tabularnewline
    Ausbildungsleiter: & Matthias Hartmann
    \tabularnewline
    Leitender Dozent: & Prof. Dr.-Ing. Olaf Herden
    \tabularnewline
    Betreuender Dozent: & Dipl.-Ing. Markus Steppacher
    \tabularnewline
    Bearbeitungszeitraum: & 08.10.2020 - 31.05.2021
    \tabularnewline
    \multicolumn{2}{l}{Freudenstadt, \today}
    \tabularnewline
\end{tabular}

\begin{flushleft}
    \newpage{}
    \par
\end{flushleft}
    \newpage
    
~

\vspace{17.1mm}

\begin{flushleft}
    \textbf{\huge{}Sperrvermerk}{\huge\par}
\par\end{flushleft}
Die vorliegende \{Projekt-, Studien-, Bachelorarbeit\} beinhaltet interne vertrauliche Informationen der Firmen Eisenmann SE und ENisco GmbH \& Co. KG. Die Weitergabe des Inhaltes der Arbeit und eventuell beiliegender Zeichnungen und Daten im Gesamten oder in Teilen ist untersagt. Es dürfen keinerlei Kopien oder Abschriften - auch in digitaler Form - gefertigt werden. Ausnahmen bedürfen der schriftlichen Genehmigung der Leitung EBZ der Firma Eisenmann SE und der ENisco GmbH \& Co. KG.
    \newpage
    
~

\vspace{17.1mm}

\begin{flushleft}
    \textbf{\huge{}Ehrenwörtliche Erklärung}{\huge\par}
\par\end{flushleft}

Ich erkläre hiermit ehrenwörtlich:

\begin{enumerate}
    \item dass ich meine Studienarbeit mit dem Thema "" ohne fremde Hilfe angefertigt habe;
    \item dass ich die Übernahme wörtlicher Zitate aus der Literatur sowie die Verwendung der Gedanken anderer Autoren an den ensprechenden Stellen innerhalb der Arbeit gekennzeichnet habe;
    \item dass ich meine Studienarbeit bei keiner anderen Prüfung vorgelegt habe;
    \item dass die eingereichte elektronische Fassung exakt mit der eingereichten schriftlichen Fassung  übereinstimmt.
\end{enumerate}

Ich bin mir bewusst, dass eine falsche Erlärung rechtliche Folgen haben wird.

%%gemäß §5 (3) der \quotedblbase Studien- und Prüfungsordnung DHBW Technik\textquotedblleft{} vom 29. September 2017. Ich versichere hiermit, dass ich meine \{Projekt-, Studien-, Bachelorarbeit\} mit dem Thema \textit{\title{}} selbstständig verfasst und keine anderen als die angegebenen Hilfsmittel benutzt habe.

\vspace{2cm}

\begin{center}
    \begin{tabular*}{\textwidth}{@{\extracolsep{\fill}}cl}
        Freudenstadt, \today & <GESCANNTE UNTERSCHRIFT>
        \tabularnewline
        & \author{}
        \tabularnewline
    \end{tabular*}
    \par
\end{center}
    \newpage
    
~

\vspace{17.1mm}

\begin{flushleft}
    \textbf{\huge{}Zusammenfassung}{\huge\par}
\par\end{flushleft}
    \newpage
    
~

\vspace{17.1mm}

\begin{flushleft}
    \textbf{\huge{}Abstract}{\huge\par}
\par\end{flushleft}


    \newpage

    \tableofcontents
    \newpage

    \pagenumbering{roman}
    %ACHTUNG: Korrekte Seitenzahl bei Fertigstellung des Dokuments einstellen, andernfalls ist die Nummerierung u.U. fehlerhaft.
    \setcounter{page}{7}

    \listoffigures
    \newpage

    \lstlistoflistings
    \newpage

    \listoftables
    \newpage

    \printglossary[title=Abkürzungsverzeichnis, type=\acronymtype]
    \printglossary
    \newpage

    %asd
    %GGF anpassen
    %Index aktualisieren mit: makeindex Vorlage.nlo -s nomencl.ist -o Vorlage.nls
    %\printnomenclature[3cm]{}
    %\newpage


    \pagenumbering{arabic}
    %Kopfzeile aktivieren
    \pagestyle{headings}
    %Alle Werke ins Literaturverzeichnis
    \nocite{*}

    %Weitere Dateien können mit \input{<Dateipfad>} eingebunden werden

\part{Erster Teil\label{part1:Erster-Teil}}
\chapter{Einleitung\label{chap1:Erstes-Kapitel}}

Einleitender Text zum Kapitel.



\section{Motivation und Problemstellung\label{sec1.1:Unterpunkt-1}}

Abschnitt 1.

\section{Vorgehen und Aufbau der Arbeit\label{sec1.2:Unterpunkt-2}}

Abschnitt 2.
\chapter{Industrielle Revolutionen\label{chap2:Zweites-Kapitel}}

Einleitender Text zum Kapitel.



\section{Erste Revolution\label{sec2.1:Unterpunkt-1}}

Abschnitt 1.

\section{Zweite Revolution\label{sec2.2:Unterpunkt-2}}

Abschnitt 2.

\section{Dritte Revolution\label{sec2.3:Unterpunkt-3}}

Abschnitt 3.

\section{Vierte Revolution\label{sec2.4:Unterpunkt-4}}

Abschnitt 4.

\subsection{IIoT\label{sub2.4.1:Unterpunkt-1}}

Unterabschnitt 1.

\subsection{Mass Customization\label{sub2.4.2:Unterpunkt-2}}

Unterabschnitt 2.

\subsection{\glqq Industrie 4.0\grqq{} in der Lehre\label{sub2.4.3:Unterpunkt-3}}

Unterabschnitt 3.
\chapter{Technologien\label{chap3:Drittes-Kapitel}}

Einleitender Text zum Kapitel.



\section{PHP\label{sec3.1:Unterpunkt-1}}

Abschnitt 1.

\section{Python\label{sec3.2:Unterpunkt-2}}

Abschnitt 2.

\section{Raspberry Pi\label{sec3.3:Unterpunkt-3}}

Abschnitt 3.

\section{MQTT\label{sec3.4:Unterpunkt-4}}

Abschnitt 4.

\section{Programmierung der Anlage\label{sec3.5:Unterpunkt-5}}

Abschnitt 5.
\chapter{Implementierung\label{chap4:Viertes-Kapitel}}

Einleitender Text zum Kapitel.



\section{Anforderungen\label{sec4.1:Unterpunkt-1}}

Abschnitt 1.

\section{Webanwendung\label{sec4.2:Unterpunkt-2}}

Abschnitt 2.

\subsection{Prozessvisualisierung\label{sub4.2.1:Unterpunkt-1}}

Unterabschnitt 1.

\subsection{Bestellung Für Anlage\label{sub4.2.2:Unterpunkt-2}}

Unterabschnitt 2.

\section{Ein-/Ausschleusen von \glqq Produkten\grqq{}\label{sec4.3:Unterpunkt-3}}

Abschnitt 3.
\chapter{Fazit\label{chap5:Fünftes-Kapitel}}

Einleitender Text zum Kapitel.



\section{Zusammenfassung\label{sec5.1:Unterpunkt-1}}

Abschnitt 1.

\section{Ausblick\label{sec5.2:Unterpunkt-2}}

Abschnitt 2.

    \newpage
    \bibliographystyle{citation/alphadin}
    \addcontentsline{toc}{chapter}{\bibname}
    \bibliography{bibtex-daten/literaturverzeichnis} 

    \appendix
    %Weitere Dateien können mit \input{<Dateipfad>} eingebunden werden

\part{Anhang}

\chapter{Erster Anhangabschnitt}

Anhang A.

\section{Unterabschnitt}

Unterabschnitt.

\subsection{Unter-unterabschnitt}

\end{document}